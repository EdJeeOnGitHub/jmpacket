\documentclass{article}

\usepackage[margin=1.15in,top=0.5in]{geometry}
\usepackage{fancyhdr,graphicx,xcolor,hyperref,wasysym}
\usepackage[misc]{ifsym}
\definecolor{ucmaroon}{rgb}{0.50,0,0}
\definecolor{alert}{HTML}{EB811B}
\hypersetup{
    pdfauthor={},               % author
    colorlinks=true,            % false: boxed links; true: colored links
    linkcolor=ucmaroon,
    urlcolor=ucmaroon           % color of external links
}

\fancypagestyle{firstpage}{%
  \fancyhf{}% Clear header/footer
  \renewcommand{\headrulewidth}{0pt}%
  \renewcommand{\footrulewidth}{1pt}%
  % Add more detail here if needed
}
\fancypagestyle{otherpages}{%
  \fancyhf{}% Clear header/footer
  \renewcommand{\headrulewidth}{1pt}%
  \renewcommand{\footrulewidth}{1pt}%
  % Add more detail here if needed
}

\AtBeginDocument{\thispagestyle{firstpage}}
\pagestyle{otherpages}

\setlength{\parindent}{0pt}
\setlength{\parskip}{2ex}

%------------------------------------------------------------------------------%
\newcommand{\fullname}{Edward Jee}
\newcommand{\myname}{Edward Jee}
\newcommand{\myDOB}{11 December 1996}
\newcommand{\mycitizenship}{UK}
\newcommand{\myphone}{+1 (857) 777-8082}
\newcommand{\myemail}{\url{edjee@uchicago.edu}}
\newcommand{\mywebsite}{\url{https://EdJeeOnGitHub.github.io/}}
%------------------------------------------------------------------------------%
\newcommand{\capfieldlist}{Development Economics, Applied Econometrics}
\newcommand{\fields}{development economics and applied econometrics}
\newcommand{\onesenID}{the dynamics and causes of persistent poverty in low-income countries}
\newcommand{\threesenID}{My primary research fields are \fields, with a focus on
understanding the dynamics of extreme-poverty as well as evaluating the effectiveness of interventions aimed at 
improving health outcomes in low-income countries. My
research agenda addresses \onesenID. To that end, I develop a new empirical framework
and harmonize microdata from randomized controlled trials to test whether poverty traps 
exist and whether targeted interventions can effectively alleviate poverty.
}
%------------------------------------------------------------------------------%

%------------------------------------------------------------------------------%
%                               teaching, by type                              %
%------------------------------------------------------------------------------%

\newcommand{\standardteach}{In my teaching, my goal is to train capable and curious thinkers who can connect
economic theory to real-world challenges. I emphasize the importance of solid 
fundamentals while encouraging students to apply what they have learnt through 
coding assignments and empirical projects. At the undergraduate and graduate levels, 
I am prepared to teach core courses in econometrics and field courses in 
development economics and applied econometrics.}
\newcommand{\generalteach}{In my teaching, my goal is to train capable and curious thinkers who can connect
economic theory to real-world challenges. I emphasize the importance of solid 
fundamentals while encouraging students to apply what they have learnt through 
coding assignments and empirical projects. At the undergraduate and graduate levels, 
I am prepared to teach core courses in econometrics and field courses in 
development economics and applied econometrics.}
\newcommand{\MBA}{
  In my teaching, my goal is to train capable and analytical thinkers who can connect
economic theory to data-driven decision making. I emphasize building strong
conceptual foundations while encouraging students to apply what they learn through
coding assignments, empirical projects, and model-based analysis. At the MBA and Ph.D.
levels, I am prepared to teach core courses in econometrics and field courses in
development economics, applied econometrics, and quantitative policy analysis.}
\newcommand{\LAC}{
In my teaching, my goal is to train capable and curious thinkers who can connect
economic theory to the real world. I emphasize strong conceptual foundations
while encouraging students to apply what they learn through hands-on empirical
projects and clear, persuasive writing. At the undergraduate level, I am prepared
to teach core courses in econometrics and field courses in development economics
and applied econometrics. I am especially drawn to teaching in a liberal arts
environment, where small class sizes and close mentorship allow students to engage
deeply with ideas and connect economic reasoning to broader social and policy
questions.}
\newcommand{\MPP}{In my teaching, my goal is to train capable and analytical thinkers who can use
economic tools to inform real-world policy decisions. I emphasize strong
conceptual foundations in economics and econometrics while encouraging students
to apply these ideas through coding assignments, data analysis, and policy
simulations. At the master's and Ph.D. levels, I am prepared to teach core
courses in econometrics and field courses in development economics, applied
econometrics, and program evaluation. My teaching highlights how rigorous
empirical methods can translate into actionable insights for policy design and
implementation.
}

%------------------------------------------------------------------------------%
%                                  job details                                 %
%------------------------------------------------------------------------------%

\input{currentjob.tex}

%------------------------------------------------------------------------------%
%                                   main text                                  %
%------------------------------------------------------------------------------%

\begin{document}

\vspace*{\dimexpr-\headsep-\headheight-1pt}

\includegraphics[width=2.5in]{uchicagologo.png}
\vspace{-6mm}

\rule{\linewidth}{1pt}

\medskip

\hfill
\begin{tabular}{ l @{} }
  \today \\[12pt] % Date
  University of Chicago\\
  Department of Economics\\
  1126 E 59 Street\\
  Chicago, IL 60637 USA\\[6pt]
  \phone~\myphone \\
  \Letter~\myemail
\end{tabular}

\medskip

\salutation

\bigskip

I am writing to apply for the position of \jobtitle~at \institution.
I expect to complete my Ph.D. in Economics at the University of Chicago in Spring 2026.
My primary advisor is Michael Kremer, and my fields of interest are \fields.
\fit

%------------------------------------------------------------------------------%

My research agenda addresses \onesenID.
% adapted JMP abstract
In my job market paper, I revisit a central question in development economics:
do poverty traps exist? Using harmonized microdata from 27 randomized controlled
trials covering 75,000 households, I develop a new empirical framework to
estimate their prevalence. I find that while 60\% of studies show signs of
poverty traps, only about a quarter of households are actually trapped---largely
due to differences in productivity and forward-looking consumption. A calibrated
growth model shows that optimal transfer policies reflect this,
targeting the poorest rather than those near the trap threshold.

% rework the transition sentence from your research statement to
In other work, I study health interventions in low-income countries. One project
examines how private costs and social image returns interact in a Kenyan
deworming campaign: as distance to treatment rises, take-up falls, but those who
deworm gain higher social image returns. A structural model shows these returns
can offset private costs, allowing policymakers to space treatment sites farther
apart while maintaining coverage. In related work, I estimate the impact of a
large-scale child immunization program in Pakistan where government programmatic
constraints limited randomization.

% In my job market paper, I address one of development economics' oldest
% questions: do poverty traps exist? I develop a new empirical framework to
% estimate the existence and prevalence of poverty traps.  By harmonizing
% microdata from randomized controlled trials covering 75,000 households, I am
% able to estimate how poor households’ assets evolve over time, accounting for
% differences in their ability and endogenous consumption-savings decisions. I find 
% that poverty traps are widespread: 60\% of studies show signs of them. Yet within a
% study, on average, only 25\% of households are actually trapped. This is driven by
% differences in productivity and forward-looking consumption responses which allow
% households to preemptively invest and escape low-asset states. Using a calibrated
% household growth model, I show that optimal transfer policy reflects this
% heterogeneity – targeting the poorest households rather than those near the trap
% threshold in a majority of contexts.
% In other work, I study the impact of health interventions on economic outcomes
% in low-income countries. One project examines how private costs and social image
% returns interact in the context of a deworming campaign in Kenya. As the
% distance to treatment increases, take-up of deworming medication declines, but
% those who do participate gain higher social image returns. Using a structural
% model, we show that these social image returns can partially offset the private
% costs of participation, implying that policymakers can optimally space treatment
% locations farther apart while still achieving broader coverage. In related work,
% I estimate the impact of a large government program in
% Pakistan that incentivized child immunizations, where programmatic constraints
% precluded an ideal randomized evaluation.


%------------------------------------------------------------------------------%

\teaching

%------------------------------------------------------------------------------%

\desire

%------------------------------------------------------------------------------%

Please find my application materials enclosed, along with recommendation letters
from Professors Michael Kremer, St\'ephane Bonhomme, \fourthLOR and
Anne Karing.
You can find updated and additional materials on my website: \mywebsite.

Thank you for taking the time to consider my application.
I look forward to hearing from you soon.

% Make sure cover letter stays on one page by adjusting the skip sizes as needed
\customCLskip

Sincerely,

\smallskip

\myname
% \fullname

\end{document}